\documentclass{article}
\author{Sam Nolan}
\title{Explain it like I'm 2: Using complements for subtracting numbers}
\usepackage[normalem]{ulem}
\usepackage{amsmath}
\begin{document}
  \maketitle

  As we have seen, there are quite easy to understand processes for adding 2 binary numbers.

  Now comes the challenge of doing the exact opposite, subtracting binary numbers.

  Lets look at how you would intuitively look at this problem.

  Consider the example below:
  \begin{align}
   &1110001 \nonumber \\
 - &0110110 \nonumber
  \end{align}

  Lets work this out like a normal decimal problem
  \begin{align}
     1110001& \nonumber \\
    -0110110& \nonumber \\
        1& \nonumber
  \end{align}
  
  So far so good
  \begin{align}
     111& \nonumber \\
    110\text{\sout{000}}&1 \nonumber \\ 
    - 11011&0 \nonumber \\
       1&1 \nonumber 
  \end{align}

  Ooh, not pretty this might make sense to us humans, but this type of carry over means that it's difficult
  for different bits to be subtracted separately within a logic circuit. If you can find a way
  to represent this type of carry, awesome! But your logic circuit is going to be quite complicated. 
  Subtraction seems to be a bit more difficult than addition.

  \section{Using the complement for subtraction}

  Using complements for subtraction is a really cool technique. I believe the best way to explain
  it is to forget about binary and see it in action within base 10.

  A complement in math is another number that adds to the original to make a number of interesting
  value. For instance, if 2 angles are complementary, they add up to 90 degrees.

  In this case, what I'll call the "10s complement" (Don't use this terminology, I invented it) of a number
  is simply another number that when added to the original, gets to a power of 10.

  For example, 27's "10's complement" is 73, as $27 + 73 = 100$.

  What power of 10 you choose here doesn't matter, as long as it adds to a number that is a power of 10
  that is larger than the two numbers that you are subtracting from each other.

  For instance, it is equally valid to say that the "10's complement" of a 27 is 973, as 27 + 973 = 1000.

  An easy way to get the "10's complement" of any number is for every digit, choose a corresponding digit
  that adds up to 9, and then add 1. For instance:

  \begin{align}
    \text{original number: } &1234567890 \nonumber \\
    \text{10's complement: } &8765432109 + 1 = 8765432110 \nonumber \\
    \text{sum: }             &9999999999 + 1 = 10000000000 \nonumber
  \end{align}

  Splitting the $+1$ is just so that you can see why it always works.

  Now, what's interesting about the complement is that it is that it is the original number \textit{away}
  from the power of 10. For instance, 27's "10's complement" is 73, and 73 is 27 below 100. The 10s complement
  is defined like this:

  \[ x + 10comp(x) = \text{power of 10} \]
  \[ x = \text{power of 10} - 10comp(x) \]
  
  Imagine a block of clay that has a height of 100cm in comparison to the ground. If there is a hole that's bottom
  is 73cm from the ground, then the hole has to be 27cm deep.

  Say we now have a block that is 50cm tall, and we put it in this hole. In comparison to the nearby ground, it is
  $50cm + 73cm = 123cm$ tall. and as the hole is 27cm deep, it must be $50cm - 27cm = 23cm$ above the block!

  So now we have represented the subtraction using addition! All we need to do is remove the additional power of 10 and we
  have the answer of 52 - 73, 23!

  \section{Moving our knowledge to binary}

  This same concept can be translated over to binary. Let's do it again:

  We will do $ 52_{10} - 27_{10} = 110100_{2} - 011011_{2}$

  Notice the trailing 0 to represent 27. Representing both numbers with the same amount of digits means that the
  power of 2 we will be using for the two's complement is larger than both numbers.

  \begin{align*}
    \text{Original number: } &011011_{2} (27_{10}) \\
    \text{2's complement: } &100100 + 1 = 100101_{2} (37_{10})
  \end{align*}
  Notice how 37 and 27 both add up to $64 (1000000_{2})$

  So we now have a block of clay that is 64cm tall, with a hole that's bottom is 37cm from the ground. Now all
  we need to do is add the block that is 52cm tall to the height of the hole and it should be 52-27 above the clay block.
  \begin{align*}
    110100& \\
 +  100101& \\
    1011001&
  \end{align*}
  
  This is the same as
  
  \begin{align*}
    52& \\
  + 37& \\
    89&
  \end{align*}

  which is 25cm above the clay block! To remove the 64cm all we need to do is remove the trailing 1 from the binary.

  011001

  And we have the answer! 25cm!

  \section{Negative answers}

  There is one last thing we need to keep in mind when doing subtraction. It's possible that we will get a negative answer.

  How do we know this? We can try it out with our "10's complement". We'll try $27 - 52$, the complement of 52 is 48.

  Now imagine a clay block that is 100cm tall, and a hole that is 52cm deep cut into the hole. The bottom of the hole is therefore 48cm
  above the ground.

  If we find the height of a 27cm block in the hole from the ground it will be $27 + 48 = 75$. The block is not above the clay block!
  It's fully inside the hole.

  You can tell it's negative because the sum is smaller than the power of 10. That is, $75 < 100$. This as starkly obvious because we
  are missing the telltale 1 in the 100's column. And this telltale sign is exactly the same in binary.

  How do we recover the answer then? Well, the answer should be the difference from the power of 10 to the answer that we got.

  Sound familiar?

  All we need to do is to find the 10's complement of 75 (25) and label it as someway "negative". The exact same logic can be applied in binary.
  
  In actual computer systems, the first bit of a byte is 1 if the number is negative and 0 if the number is positive. If the rest of the number
  is negative, it is stored as the 2's complement because adding the negative number to any other number is convenient that way for computers.
  However, reading the value of the 2's complement is difficult for humans so that's why we convert it back (by reapplying 2's complement)

  An interesting repercussion of this representation is that if a number is stored with only 8 bits, 128 + 1 becomes -127. If you
  want to see a real world problem caused by this, check out the Year 2038 problem.

  Hope that made sense! If it didn't, I would love some feedback. My email is s3723315@student.rmit.edu.au.
  
  If there is anything in this course of other courses in computer science that need a decent explanation, just ask! Writing these
  helps give me a good understanding of the course material.


\end{document}
