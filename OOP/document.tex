\documentclass{article}
\author{Sam Nolan}
\title{Explain it like I'm 2: Object Orientated Programming}
\usepackage{minted}

\begin{document}
  \maketitle

  This paper contains an explanation of object orientated principles as well
  as some general tips for creating clean code. It requires that you have
  some knowledge of the Java programming language.

  Writing good object orientated code is a massive topic covered by many books.
  Books I would recommend on writing good code is Clean Code by Robert C Martin and
  Code complete 2 by Steve McConnell. For something more advanced, try Design patterns
  (which is colloquially known as GOF)

  So how does Object Orientated Programming work?

  \subsection*{Classes}
  Almost all OOP languages use classes. Some would argue that if a language
  claims to be OOP and does not have classes (For instance, early JS and Lua)
  that they are not really OOP, and there is definitely a case for that.
  Classes are at the core of OOP.

  A class is simply a description of an object. There are two main things in
  this description. The instance variables, which are simply variables that are
  related to an object. This is what the object \textbf{has}. The other thing
  in the description are the methods to the object. These are all the functions
  that are related to an object and is therefore what the object \textbf{does}.

  After we have described the object with a class, we can then use the class
  to \textbf{instantiate} the object. Instantiating uses the new keyword in
  Java.

  \begin{minted}{java}
    Dog dog = new Dog("Bojo");
  \end{minted}

  This is telling Java to create a variable of type \texttt{Dog} named \texttt{dog} which is
  created from the description of \texttt{Dog} with parameter \texttt{Bojo} (which I would assume
  is the name).

  We can then reference things about the dog using dot notation. Such as:

  \begin{minted}{java}
    dog.bark();
    String name = dog.getName();
  \end{minted}

  \section*{Object Orientated design}
  Object orientated design is similar but different to the design we encounter
  everyday.

  The goal of Object Orientated design is maintainability. That is, how easy is
  it to add new features to our program if our code is implemented in a particular
  way. If it's not given too much thought, maintainability will decrease as
  you keep working on the project.

  It's very common to use a building analogy for software engineering, in fact,
  we are often called \textit{software engineers} and \textit{software architects} for that
  reason. If the design of the house is not given too much thought beforehand,
  and error that have been made have not been fixed, it will become harder and
  harder to build on the house. Any change you make might make it tumble down
  to the ground.

  Keep in mind that a building can look good and work fine (have a good external
  design) but still have a very fragile internal structure (have a bad internal
  design). Object Orientated design has nothing to do with the user experience
  and is instead to do with how easy it is to change.

  This is very much like software as it gets built. At first, it is very easy
  to get started and build what you need. But if you don't consider the design
  of the internal structure and just keep building, it will become difficult
  to add to it without breaking it.

  Unlike buildings, software design happens throughout the development process.
  Although it is always a good idea to do some design beforehand, it is also
  a good idea to improve the design of existing code. The process of improving
  the design of existing code is called \textbf{refactoring}.

  \subsection*{Duplicate code}
  Duplicate code is big no-no in any software project. Why? Having duplicate
  code makes it harder to make changes to the code, as you may have to make it
  in 2, 3 or maybe even more places.
  
  Duplicating code is one of the easiest ways to make your project not maintainable.
  It can also introduce subtle bugs where you remembered to update one part of
  the code but not the other part.

  In general, if you feel like you need to copy and paste, you should think
  about your design and how you can prevent the duplication.

  In the easiest case, duplication is often avoided by simple placing the code
  in a method and having the two places calling it.

  \textbf{Original}
  \begin{minted}{java}
  private Person[] people = new Person[COMPANY_SIZE];

  public void addEmployee(String name, double wage){
    Person employee = new Person(name);
    employee.setWage(wage);
    
    for(int i = 0; i < people.length; i++){
      if(people[i] == null){
        people[i] = employee;
      }
    }
  }

  public void addContractor(String name, double rate, int hoursNeeded){
    Person contractor = new Person(name);
    contractor.setRate(rate);
    contractor.employFor(hoursNeeded);
    
    for(int i = 0; i < people.length; i++){
      if(people[i] == null){
        people[i] = contractor;
      }
    }
  }
  \end{minted}

  \textbf{Refactored}
  \begin{minted}{java}
  private Person[] people = new Person[COMPANY_SIZE];

  public void addEmployee(String name, double wage){
    Person employee = new Person(name);
    employee.setWage(wage);
    
    addPersonToCompany(employee);
  }

  public void addContractor(String name, double rate, int hoursNeeded){
    Person contractor = new Person(name);
    contractor.setRate(rate);
    contractor.employFor(hoursNeeded);
    
    addPersonToCompany(contractor);
  }

  private void addPersonToCompany(Person person){
    for(int i = 0; i < people.length; i++){
      if(people[i] == null){
        people[i] = person;
      }
    }
  }
  \end{minted}

  \subsection*{Encapsulation}
  In regards to OOP, to be a valuable member of society, a class should both have something and
  do something. It must make use of it's resources wisely and not be too
  concerned in the affairs of others.

  Here is an example of a Vector class (a vector for this can be thought of as
  a point in 2D space). This vector class is not very responsible, and uses
  the VectorCalculator class to do most of its calculations.

  \begin{minted}{java}
class Vector{
  public double x;
  public double y;
}

class VectorCalculator{
  public double distance(Vector vector){
    return Math.sqrt(vector.x * vector.x + vector.y * vector.y);
  }
}
  \end{minted}

  The vector class is not considered a valuable member of society because although
  it has information it doesn't do anything with it. It requires other people
  to do things for it.

  This is analogous to buying an iPhone but when you open the box you realise
  that you just got shipped the components. The entire iPhone has been
  disassembled and all of it's internal characteristics exposed. You could
  technically use the iPhone by building it from scratch, but your not an iPhone
  builder! That was their job! What if you get it wrong? You might even end up with an iPhone that
  is slightly different from a normal iPhone and misses every second call, and
  you just wouldn't know!

  Although this Vector class example is a small one, as you get larger programs,
  this becomes a large issue. This is a classic
  theme in software development. It might be quicker and easier to write now
  but if you don't write good code now it will be hard to add new features later.

  Back to our iPhone analogy, to solve the issue we create a case for the iPhone.
  We dictate exactly how you are allowed to interact with the iPhone and what
  you can and can't do. This doesn't take power from the iPhone user, but it
  makes their lives exponentially easier.

  In programming, we call this case creating \textbf{encapsulation} we put our classes
  is neat cases that dictate what you can do and how you do it. To put a case
  on our Vector class, we would create the following:

  \begin{minted}{java}
class Vector{
  private double x;
  private double y;
  
  public Vector(double x, double y){
    this.x = x;
    this.y = y;
  }

  public double distance(){
    return Math.sqrt(this.x * this.x + this.y * this.y);
  }
}
  \end{minted}

  This vector class now has a case. We use the word \texttt{private} to not allow accessing
  the x and y instance variables from outside the class. This is called information
  hiding and is like shielding the circuitry of a phone from view, and then we
  dictate the public methods that operate on the instance variables, which are
  like the buttons and screens of our phone.

  This class is easier to understand than the former with their elements exposed
  for the same reason that the phone is easier to understand when it's got a case
  around it. The principles we use in designing products can often be applied
  to designing classes, because they are in a sense internal products to other
  classes.
  
  It is always a good practice to make all of the instance variables private.
  Why? If we go back to our phone analogy, if Apple after shipping their phones
  in pieces decided that they would like to upgrade the CPU of their disassembled
  phones, then it is likely that they would now upset customers who relied on
  the particular architecture or size of the CPU. By exposing all the internal
  components, you are encouraging people to be dependant on parts of the system
  that really should be considered none of the user's business.

  By hiding the CPU and circuitry of the phone, Apple is able to change those
  features to remain competitive in the market without upsetting customers.

  In a class sense, if we go back to our Vector class, by making the instance
  variables public we are encouraging people to become dependant on the fact
  that we are representing a Vector using an x and a y coordinate. However,
  we might have found that many people were using the distance operation often
  (which calculates the distance from the point (0, 0)), so thought that they
  should change the internal representation to polar form. Which would increase
  performance for this particular case.
  
  Using x and y is
  called Cartesian form, meaning that a point is represented by `walk this far
  in the x direction, then walk this far in the y direction and you should 
  be at the point I'm talking about' Polar form instead uses an angle and a distance.
  It says `Face towards this particular direction and walk r meters'. r is the
  distance we are calculating so in polar form we no longer need to calculate
  it. Here is the vector class using polar form

  \begin{minted}{java}
class Vector{
  private double r;
  private double theta;

  public Vector(double x, double y){
    // Calculates the direction to (x, y)
    theta = Math.atan2(y, x);

    // Calculates the distance to (x, y)
    r = Math.sqrt(x * x + y * y);
  }

  public double distance(){
    return r;
  }
}
  \end{minted}

  Now, any user of Vector wouldn't even notice the difference, except all of
  a sudden the distance operation would have got much faster.

  This wouldn't have been possible if we made the x and y public. As we would
  have had to change every point for which the class is used, which might
  be scattered throughout the entire program!

  You should always make all the instance variables of a class private. Exceptions
  to this rule are so extraordinarily few that I can't name any.

  Also, you don't want to have too many public methods on your class. This is
  analogous to a phone with too many buttons and sensors. It simply becomes
  hard to use!

  \subsection*{Interfaces}

  One thing that's very important in encapsulation is that the users of the
  class should become dependant on the interface of the class, and not it's
  implementation. That is, apple users should become dependant on the graphical
  interface and not the CPU architecture, and our Vector class users should
  be dependant on the things it does rather than how it's represented.

  It's often useful to see what a class looks like from the outside. From the
  outside, our Vector class looks like this:

  \begin{minted}{java}
    public double distance();
  \end{minted}
  
  Other than the constructor, This is what we can see, the externals of the class.

  We can call this the class interface. If we would like to, we can make this
  interface explicit by creating an interface in java.

  \begin{minted}{java}
interface Vector{
  public double distance();
}
  \end{minted}

  And then we can say that our polar form and Cartesian vectors can \textbf{implement}
  this interface.

  \begin{minted}{java}
class PolarVector implements Vector {
    ...
}

class CartesianVector implements Vector {
    ...
}
  \end{minted}

  This way we completely separate the interface from it's implementation. We
  could keep creating new classes and as long as it implements the interface
  we can interact with it.
  
  What's great about using interfaces is that it becomes extraordinarily easy
  to swap out different implementations. Say we define our Cartesian vectors
  like so:

  \begin{minted}{java}
Vector vector = new CartesianVector(x, y);
  \end{minted}

  If every time we reference a vector we use the Vector interface as a type,
  and you want to change the implementation of the vector, all you need to do
  is this:

  \begin{minted}{java}
Vector vector = new PolarVector(x, y);
  \end{minted}

  Only the point where it's instantiated needs to change! Not even the type,
  the type can just remain as Vector.

  Interfaces add flexibility into your program, they make it easy to swap out
  one implementation for another. So if you perceive that it might be required
  to swap out different implementations you should use an interface.

  \subsection*{Abstraction}
  Now that we know that it's better to be dependant on an interface rather 
  than an implementation, what makes a good interface? That's the question of
  abstraction. An abstraction is a metaphor to how the system actually works
  in order to make our understanding easier.

  One example of an abstraction is a file system, data is not actually stored in folders and files, on
  a computer it's either stored on a metal platter or a computer chip.
  The file system is an abstraction of what's actually happening, and allows
  the user to interface with the system without needing to understand exactly
  how the system works.

  Computer systems are quite famous for having layers and layers of abstraction.
  When you create an account on a website, you are either:

  \begin{enumerate}
    \item Creating an account
    \item Pressing a button on the web page
    \item Sending a POST request to a web server
    \item Creating a record in a database
    \item Saving your details to a file
    \item Saving your details to a hard disk
  \end{enumerate}

  Every time we add a layer of abstraction, the process becomes easier to use
  but more specialised.

  This is very similar to how human language works. We start off by understanding
  very simple terms like `car' and `cat', and then we use those concepts to provide
  larger concepts such as `traffic' and `species'. Our understanding keeps building
  on what we already know.

  The aim of this language job is to as succinctly as possible express what you
  are trying to express with as few words as possible. In programming, it's very
  similar, the high level abstractions talk in the language of the problem,
  and the low level abstractions talk in the language of the implementation.

  When it comes to building classes, we have classes that have a high level
  of abstraction such as:

  \begin{minted}{java}
    UserAccount account = new UserAccount("username", "password", "display name");
  \end{minted}

  Or we could have something that is at a low level of abstraction such as
  \begin{minted}{java}
    File file = new File("filename");
  \end{minted}

  The things that are at the high level of abstraction are simply a collection
  of the things at the lower level, but they are much easier to use.

  Our vector class is an abstraction for 2 floats.

  When creating a class, you should consider what level of abstraction your
  class is, and sticking to that level. For instance, a class with the following
  interface:

  \begin{minted}{java}
  class AccountList{
    public void addAccount(Account account);
    public Account getAccountById(String id);
    ...
    // More account related features
    ...
    public void writeToFile(String filename, byte[] contents);
  }
  \end{minted}
  
  Violates the abstraction because of the \texttt{writeToFile} method. This method is
  on a much lower level of abstraction than the rest of them, and as of such
  should not be included in this class.

  \subsection*{Abstract classes}
  An abstract class is a mixture between a class and an interface. An abstract
  class has some methods that are implemented but then also provides an interface
  to implement other methods.

  As there are some methods that are left unimplemented, it cannot be instantiated
  unless a subclass of an abstract class gives an implementation for them.

  Here is an example of a shape abstract class with subclasses
  
  \begin{minted}{java}
abstract class Shape{
  private String name;
  public Shape(String name){
    this.name = name;
  }

  public String getName(){
    return name;
  }

  public abstract double getArea();
}

class Square extends Shape{
  private double sideLength;

  public Square(double sideLength){
    super("Square");
    this.sideLength = sideLength;
  }

  public double getArea(){
    return sideLength * sideLength
  }
}

class Circle extends Shape{
  private double radius;

  public Circle(double radius){
    super("Circle");
    this.radius = radius;
  }

  public double getArea(){
    return Math.pi * radius * radius;
  }
}
  \end{minted}

  The Shape class here is abstract, as it has the feature of remembering it's
  name, which is a common implementation among all shapes, but also says that
  it's children need to define a \texttt{getArea()} method.

  the \texttt{getArea()} method works like an interface, and the \texttt{getName()} works like a
  normal class.

  An abstract class cannot be instantiated, you cannot call \texttt{new Shape("Squircle")}
  as it will come up with a compiler error. This is because it does not yet know
  how to implement the \texttt{getArea} method without subclassing.

  Abstract classes are useful for mixing classes with interfaces, as well as having
  a superclass call a method of a subclass, for instance, you could add to your Shape
  class the following method:

  \begin{minted}{java}
  public String getDetails(){
    return "Area: " + String.valueOf(getArea());
  }
  \end{minted}

  This would get the value of the area from whatever implementation the shape
  and return it in a standard format.

  It only makes sense to make a class abstract if it has subclasses. For our
  Shape example, the question to ask is `Is it useful to have generic Shapes
  in our program?' If not, make the class abstract, if so, keep it concrete
  (opposite of abstract)

  \subsection*{Single responsibility principle}
  There is so much in object orientated design that we could touch on, but I'll
  finish this paper with a very important concept popularised by Robert C Martin
  in his book Clean Code.

  This important principle is that every method or class should do one thing,
  and exactly one thing.

  If you could argue that the function you are making does two things or has two
  steps. You should split one step them into private function and call them 
  from the first function.

  This ensures that every method is small and easy to understand. It makes it
  an absolute breeze to find errors in a method that's only 10 lines long.

  Furthermore, each class should do exactly one thing. Or in Robert's words,
  `A class should have only one reason to change'. If you could easily split
  the class into different things that it does or represents, you should consider creating two or more
  classes for the different features.

  In practical terms, it's a good idea to keep methods below 30 lines, there's
  no practical limit for classes, but in well written frameworks the average class
  size is around 80 lines.

  If you want to see more examples on how to write clean code, I would recommend
  the books mentioned in the introduction.

  I hope that this was useful! If you have any feedback, please let me know at
  s3723315@student.rmit.edu.au or check out the GitHub repository for these
  papers at github.com/Hazelfire/EILI2. If you feel like there's a topic in
  computer science that needs a good explanation, just let me know!
\end{document}
