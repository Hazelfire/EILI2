\documentclass{article}
\author{Sam Nolan}
\title{Explain it like I'm 2: RSA}
\usepackage{amsmath}

\newcommand{\inmod}{\ \text{mod}\ }

\begin{document}
  \maketitle

  The encryption of information is a difficult task. You have to transform data in such a way that it cannot be transformed back without a key, and also does not give away any information about the message.

  RSA is one such algorithm. It's special because it is an asymmetric cryptosystem. Meaning that there is one key that encrypts the information, and the other key decrypt it.
  
  It does this through modular arithmetic and powers. To explain RSA, we will have to look into the mathematical concepts that it is based off. If you don't feel great about maths, don't worry! We'll go through all the concepts.

  \subsection*{Power laws}
  RSA makes use of this simple power law:

  \[ (a^{b})^c = a^{bc} \]

  Try it out for yourself! Punch in a few numbers.

  If you're still not convinced, let's think about it for a second, $(a^b)^c$ means that there are $c$ groups of $(a^b)$ multiplied together, and $a^b$ means that there is $b$ copies of $a$ that are multiplied together. The amount of $a$ in this expression is $bc$, which shows that the expression above is true.

  That's all for exponents that we'll need, now for the fun part!

  \subsection*{Modular Arithmetic}
  Sounds horrifying! It isn't that bad, we use modular arithmetic every day.

  The clock is a form of modular arithmetic. This is because if it is 11 o'clock and you wait for 2 hours, it will be 1 o'clock. This tick over is what makes time modular.

  This is expressed as the following:
  \[ (11 + 2) \inmod 12 = 1 \]

  This $\inmod$ is an operation in the same way that addition and multiplication is. $\mod 12$ gets rid of any spare 12's in the 13. An easy way to get rid of these spare 12s is to divide the number by 12 and find the remainder. This operation is represented by a \% in Java.

  There is another notation that this can be expressed as.
  \[ 11 + 2 \equiv 1 \mod 12 \]

  The $\equiv$ sign is a telltale sign that something is different. The $\equiv$ sign is read as "is equivalent to", and the entire expression reads as "11 + 2 is equivalent to 1 under mod 12". This is saying that if we are using modular arithmetic with base 12, then $11 + 2$ and $1$ are equivalent.

  The \mod 12 is in a sense a marker telling you to treat the following
  relationship as modular. It's like saying, "If we are talking about clocks, 
  then 11 + 2 is equivalent to 1" The $\inmod 12$ part of this equation is 
  stating something about the entire equation, and not just the 1.

  If you don't like thinking about different systems, the statement is basically saying:
  \[ 11 + 2 \inmod 12 = 1 \inmod 12 \]

  The only difference between clocks and modular arithmetic is that modular arithmetic starts at 0, for instance.

  \[ 12 \inmod 2 = 0 \]

  I like modular arithmetic! Imagine if there was only 12 numbers in existence and they just kept looping around! No more massive sums!

  To top it off, when using modular arithmetic, there is no such thing as decimals, negatives or division! Only whole numbers.

  Working with modular arithmetic algebreically works very similar to working with
  normal numbers. For instance, multiplying both sides of the equation by the same
  number will still create the same number. The only rule of thumb is that if
  it's an operation that can make a fraction or decimal out of a whole number,
  then it's banned from this point onwards. 
  
  What I mean is because dividing by 2 can create half a number when dividing
  by an odd number, it is therefore banned. That means:

  \begin{itemize}
    \item No surds
    \item No fractions or decimals
    \item No logarithms
    \item No division
    \item No negatives
  \end{itemize}
  \subsection*{Coprimes and our least favourite times table}
  
  Which was your least favourite times table? In my opinion, 7 wins the prize all the way. With seemingly no pattern between the numbers.

  There are some times tables that have patterns with the last digit. For instance, our beloved 5 times tables, which can only end in 0 or 5 and makes our life much easier.

  Let's look at our 7 times tables last digits, but with a new perspective. To use modular arithmetic, $\inmod 10$ will just get the last digit of our number. This is because $\mod 10$ gets rid of any excess 10s, which also includes any 100s (because 100s are made of 10s)

  So, we have the following:
  \begin{align}
    7 \times 1 \equiv 7 \mod 10 \\
    7 \times 2 \equiv 4 \mod 10 \\
    7 \times 3 \equiv 1 \mod 10 \\
    7 \times 4 \equiv 8 \mod 10 \\
    7 \times 5 \equiv 5 \mod 10 \\
    7 \times 6 \equiv 2 \mod 10 \\
    7 \times 7 \equiv 9 \mod 10 \\
    7 \times 8 \equiv 6 \mod 10 \\
    7 \times 9 \equiv 3 \mod 10 \\
    7 \times 10 \equiv 0 \mod 10 
  \end{align}
  Yuck. But there's something cool about this, if you multiply 7 by a number under mod 10, \textbf{any} number as a last digit is possible. Count them above!

  This occurs when the mod we are using is \textbf{coprime} to the number we are multiplying. A number is coprime when there is no common factors between the numbers other than 1. Try it out! The numbers that are coprime to 10 are 1, 3, 7 and 9, all of which can have any digit as a last one.
  
  This property comes in useful for our next part.

  \subsection*{Multiplicative Modular Inverses}
  My god that sound scary. It's not really, it's similar to all the other inverses
  we have learnt in primary school.

  What is an inverse? If I apply an operation to a number, the inverse of that
  operation reverses the original. For instance, the inverse of $+1$ is $-1$
  because if you have any number, then add 1, and take 1, you get back the same
  number.

  A multiplicative inverse is an operation that reverses the multiplication
  operation. In normal real numbers, if you multiply a number by $x$, to get the
  original number, you divide by $x$. This can also be represented by multiplying
  by $\frac{1}{x}$ or $x^{-1}$. With real numbers, $\frac{1}{x}$ and $x^{-1}$ are exactly the same,
  just different ways of writing it.

  However, this is where the "Modular" part comes in. Because as we noted before,
  in modular arithmetic, there is no such thing as division. So what are we to do?

  Well, unlike real numbers, when you multiply under modular arithmetic, numbers
  can actually get smaller. For instance:

  \[ 7 \times 2 \equiv 4 \mod 10 \]

  Here, by multiplying by 2, we go from 7 to 4. That's special to modular arithmetic.
  It means that we can get back to where we started by multiplying rather than
  dividing.

  So say we are trying to find a modular multiplicative inverse of 7 for mod 10.
  This means that under mod 10, if I multiply a number by 7, and then multiply it
  by the modular multiplicative inverse of 7, we should get the original number.

  This is stated mathematically as the following:

  \[ x \times 7 \times 7^{-1} \equiv x \mod 10 \]

  Note here my notation for the inverse of 7 as $7^{-1}$. For those who are in
  to maths, this no longer means $\frac{1}{7}$, and instead means the modular
  multiplicative inverse of 7 for mod 10. This is what is called an abuse of
  notation, meaning that what is stated is not formally correct and can be
  misleading, but is a very common form of notation and is generally accepted.

  What's special about multiplicative inverses in real numbers is that if I multiply
  a number $x$ by it's multiplicative inverse I'll call $x^{-1}$. Then $x \times x^{-1} = 1$.

  In this case you can see that $7 \times 7^{-1} \equiv 1 \mod 10$. As
  Any number $x$ times 1 is equal to $x$. So all we need to find the number that
  multiplied by 7 gives 1 under mod 10. This number happens to be 3. So we
  therefore say that 3 is the multiplicative inverse of 7, or with maths:

  \[ 7^{-1} \inmod 10 = 3 \]

  This means that under mod 10, if we are to multiply a number by 7, and then by
  3, we'll get the same number again. Try it out!

  One interesting thing to note is that the inverse of 7 mod 10 is 3, and also
  the opposite is true, that is, the inverse of 3 mod 10 is 7. This is true
  because it doesn't matter what order you multiply the numbers.

  \[ 3^{-1} \inmod 10 = 7 \]
  \[ 7^{-1} \inmod 10 = 3 \]

  Finally, it is only possible to have a modular multiplicative inverse if the
  number that you are multiplying by is coprime to the modular. As we saw before,
  multiplying 7 under mod 10 can get any numbers 0-9, which means that it's 
  guaranteed to be able to find one combination that gives a 1. This only works
  if the number is coprime to the mod. For instance, you won't find a number where:

  \[ 8 \times x \equiv 1 \mod 10 \]

  Because 8 is not coprime to 10.

  \subsection*{RSA, key generation}
  Now for the moment of truth, now that we've looked at all the previous topics,
  we'll be able to tackle RSA head on.

  First of all, we'll look at key generation. Using 2 prime numbers $p$ and $q$
  we are able to generate prime number $n$ by multiplying them, and then we get
  $\phi(n)$, which we use to generate $e$ and $d$.

  For those who are in to maths, the $\phi(n)$ is actually \textit{Euler's Totient
  Function}. Which, for when n is made up of 2 primes, $\phi(n)$ happens to be
  $(p-1)(q-1)$. You can look up the actual definition of $\phi(n)$ if you're
  interested.

  So now that we have generated $n$ and $\phi(n)$, we generate $e$ and $d$.
  First we choose $e$ to be coprime to $\phi(n)$. Why? Because it makes the
  modular multiplicative inverse of $e$ possible. So we then choose $d$ to be
  the modular multiplicative inverse of $e$.

  The only property that we actually care about is that $e$ and $d$ are multiplicative
  modular inverses of each other over $\phi(n)$. That is:

  \[ d \times e \equiv 1 \mod \phi(n) \]
  
  If you choose any two numbers for $d$ and $e$ that follow this rule, RSA will
  work. It might not be secure, but it will work. For instance, choosing 1 for
  both $d$ and $e$ would create a system where no encryption or decryption will
  ever happen. As raising the message to the power of 1 will do nothing to it.
  The encryption and decryption steps will work, but it will not be secure.

  Take note that we can generate these the other way around. Find a number $d$
  that is comprime to $\phi(n)$ and then find $e$. $d$ and $e$ are in a sense
  equivalent, of equal status. Why $e$ is usually chosen before $d$ in practice
  is outlined later, but in theory, they are interchangeable.

  $d$ is the modular multiplicative inverse of $e$ and vice versa.

  \subsection*{The Encryption and Decryption}

  Now to the heart of it! Here we show you why RSA seems to magically encrypt
  and decrypt numbers.

  \[ M^e \equiv C \mod n \]

  Where $M$ is the message and $C$ is the cipher text.

  The message is then decrypted by:

  \[ C^d \equiv M \mod n \]
  
  To show this is true, I'll substitute the definition of $C$ in the encryption
  step into the decryption step. So that:

  \[ (M^e)^d \equiv M \mod n \]

  Now if we remember from our first section, this is the same as

  \[ M^{ed} \equiv M \mod n \]

  Now, as $e$ is the modular multiplicative inverse of $d$, as $d \times e = 1$,
  doing this operation cancels out the $d$ and the $e$ so

  \[ M^{1} \equiv M \mod n \]

  And there we have it! Taking a message to the power of $e$ encrypts the message,
  and $d$ simply cancels out $e$ because it's the multiplicative inverse!

  Something to take away from this is that because $e$ and $d$ are generated 
  the same way, and are inverses of each other, you can change this around and
  have the server encrypt a message with $d$ and let anyone decrypt it with $e$.
  This is called digital signing and is simply the reverse of encryption, we'll
  cover it later.

  \subsection*{$\phi(n)$ and the real story}
  OK, I lied to you, there's a flaw in my logic above. But the way that it's 
  illustrated allows you to understand what you can do with RSA, helps you
  remember the encryption and decryption steps and gives you an idea of where
  digital signing comes from. The truth is a bit more ugly.

  If you're not a maths person, skip this section, because it doesn't help to
  explain anything and doesn't give much more understanding of RSA. If you're
  a maths person, read on!

  So, you may have noticed that $d \times e = 1$ is not true, and the real story
  is:

  \[ d \times e \equiv 1 \mod \phi(n) \]

  And we are not doing a mod under $\phi(n)$, saying that $d \times e = 1$ is
  not true. The whole starts here:

  \[ M^{ed} \equiv M \mod n \]

  Given the definition of $d \times e$ above, it tells us that $d \times e$ is
  some multiple of $\phi(n)$ plus $1$. That is, $d \times e$ can be written as:

  \[ 1 + a\phi(n) \]
  
  Where $a$ is some arbitrary natural (whole) number. Let's substitute this in for $ed$

  \[ M^{1 + a\phi(n)} \equiv M \mod n \]

  If you know your exponent laws well, you can split up exponents that have a 
  plus in it, so:

  \[ M^{1} \times M^{a\phi(n)} \equiv M \mod n \]

  Now I can get rid of the M on either side and we get:

  \[ M^{n\phi(n)} \equiv 1 \mod n \]

  Now, If $M^{\phi(n)}$ is $1$ under $\inmod n$ then that would mean that any 
  power of it would also be one. So if the following is true, the above is true

  \[ M^{\phi(n)} \equiv 1 \mod n \]

  I'm going to stop here, as the above is a theorem by the name of 
  \textit{Euler's Theorem}, which is a bit more complicated. But because this
  is true, the above is all true, and RSA works.

  \subsection*{Digital Signing}
  You can imagine that taking a number to the power of $e$ times the exponent by
  $e$ and then taking it the power of $d$ divides the exponent by $e$ and gets
  back to $1$.

  $e$ is like the number 2, and $d$ is like $\frac{1}{2}$. If you multiply by 2, you
  can reverse it by multiplying by $\frac{1}{2}$. But the reverse is also true,
  if you multiply by $\frac{1}{2}$, it can be reversed by multiplying by $2$.

  In a real life situation, that means that data can be "Encrypted" in a way that
  every can unlock with the public key. What's the point of that?

  It actually comes in very useful when we are signing documents. By being able
  to encrypt a small document (such as a hash) that is decryptable by everyone,
  you are able to prove your identity.

  How this works is that you hash the document you are sending, encrypt the hash
  with your private key, and then others can read the decrypt the hash and 
  verify that what you sent was what you wanted to send, and that it actually
  was sent by you.

  Digital signing is simply encryption in reverse, using the private key to
  encrypt and the public key to decrypt

  \subsection*{RSA in practice, $e$ and $d$}
  In this section I'll talk about what is done to choose $e$ and $d$ in practice,
  and why $e$ is chosen first.

  It's a very simple reason really, it's because we cheat. We usually just
  choose $e$ to be $65537$ ($2^{16} + 1$), which is prime, so will therefore by coprime to
  almost any number.

  Keep in mind that this doesn't mean the everyone's public key is the same,
  because the public key is also made of $n$, which will be different for
  every key.
  
  What's also cool about $2^{16} + 1$ is that it's got a lot of binary $0$s in it
  ($10000000000000001_2$), which actually makes exponentiation easier in practice.

  We then derive $d$ from $e$ and $\phi(n)$. That's the only reason why we generate
  $e$ before $d$.

  I hope that made sense. If you have any feedback, or found any errors, these
  papers are now on Github! at \texttt{github.com/Hazelfire/EILI2}. You can
  submit an issue or just simply send me feedback at \texttt{s3723315@student.rmit.edu.au}.

  If you have any other topics that you think need a decent explanation, send
  me an email and I'll try and write one up! These papers really help me with
  my understanding of the course material too. So I'd be more than happy to
  write a few more.
\end{document}


